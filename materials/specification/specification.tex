\documentclass[9pt]{article}
\usepackage[margin=0.75in]{geometry}
\usepackage[T1]{fontenc}
\usepackage[polish]{babel}
\usepackage[utf8]{inputenc}
\usepackage{hyperref}

\title{Implementacja gry Sokoban w C i GTK}
\author{Igor Krzywda}
\date{\today}

\begin{document}

\maketitle

\section*{Zasady Sobokan}
Cel gry polega na rozmieszczeniu skrzyń w odpowiednie miejsca w magazynie. Magazyn
jest planszą składającą się z kwadratowych pól, na których mogą się znajdować
ściany, skrzynie lub gracz.
Skrzynie są przemieszczane przez postać kontrolowaną przez gracza według 
następujących zasad:
\begin{itemize}
    \item postać może poruszać się jedynie po pustych polach w 4 kierunkach:
        lewo, prawo, góra, dół
    \item skrzynie można jedynie pchać
    \item nie da się pchnąć jednej skrzyni za pomocą drugiej
    \item ściany są nieruchome
\end{itemize}

\section*{Interakcja z użytkownikiem}

\subsection*{Menu główne}
Menu główne będzie się składało z planszy 5 na 4 wypełnionej przyciskami
ponumerowanymi od 1 do 20 reprezentującymi poziomy. Aby rozpocząć rozgrywkę
należy kliknąć w przycisk z wybraną liczbą, który rozpocznie grę na danym 
poziomie.

\subsection*{Rozgrywka}
Podczas rozgrywki gracz będzie kontrolował postać strzałkami. 
Dodatkowo będzie miał opcje cofnięcia ruchu za pomocą \texttt{`u'} oraz 
zapauzowania gry z użyciem \texttt{`ESC'}. Czas gry będzie stale wyświetlany w 
lewym górnym rogu.
Po rozmieszczeniu poprawnie wszystkich skrzyń gra automatycznie się kończy i 
przenosi gracza do ekranu końcowego.

\subsection*{Koniec rozgrywki}
Po ukończeniu poziomu zostaje wyświetlony czas wraz z informacją czy jest to nowy
rekord. Dostępne będą opcje powtórzenia poziomu lub powrotu do menu głównego pod
odpowiednimi przyciskami.

\section*{Implementacja}
Program będzie podzielony na trzy główne moduły. Moduł \texttt{sobokan} będzie
abstrakcyjną reprezentacją gry, gdzie planszą będzie tablica dwuwymiarowa
znaków reprezentujących odpowiednie obiekty. Za pomocą sygnałów będzie można 
zmieniać stan planszy (np.ruch do góry będzie zmieniał stan pola, na którym
znajduje się postać i pola bezpośrednio nad nią w zależności od ich relacji). Drugi 
moduł \texttt{game} będzie 
obsługiwał główną pętlę gry czyli wyświetlanie planszy z użyciem tekstur przy
pomocy biblioteki GTK oraz czytanie i wykonywanie instrukcji gracza. 
Ostatni moduł \texttt{gui} będzie odpowiedzialny za interakcje z użytkownikiem 
na poziomie wszystkich menu.

\section*{Źródła}
\begin{itemize}
    \item ``Sobokan''. Wikipedia, Wikipedia Foundation, 7 listopada 2021, 
        \url{https://en.wikipedia.org/wiki/Sokoban}
\end{itemize}

\end{document}
