\documentclass[9pt]{article}
\usepackage[margin=0.75in]{geometry}

\title{Implementacja gry Sokoban w C i GTK+}
\author{Igor Krzywda}
\date{\today}

\begin{document}

\maketitle

\section{Sokoban}

Gra polega na rozmieszczeniu skrzyń w odpowiednie miejsca w magazynie.
Skrzynie są przemieszczane przez postać kontrolowaną przez gracza według 
następujących zasad:
\begin{itemize}
    \item skrzynie można jedynie pchać
    \item nie można pchnąć jednej skrzyni za pomocą drugiej
    \item ściany są nieruchome
\end{itemize}

\section{Interakcja z użytkownikiem}

\subsection{Menu główne}
Menu główne będzie się składało z planszy 5 na 4 wypełnionej kwadratami 
ponumerowanymi od 1 do 20 reprezentującymi poziomy. Aby rozpocząć rozgrywkę
należy kliknąć w wybrany kwadrat.

\subsection{Rozgrywka}
Podczas rozgrywki gracz będzie kontrolował postać strzałkami lub przyciskami 
`hjkl'. Dodatkowo będzie miał opcje cofnięcia ruchu `u' oraz zapauzowania gry 
`ESC'. Czas gry będzie stale wyświetlany w lewym górnym rogu.
Po rozmieszczeniu poprawnie skrzyń gra automatycznie się kończy.

\subsection{Koniec rozgrywki}
Po ukończeniu poziomu zostaje wyświetlony czas wraz z informacją czy jest to nowy
rekord. Dostępne będą opcje powtórzenia poziomu lub powrotu do menu głównego.

\begin{itemize}
    \item najlepszy czas dla każdej trudności
    \item liczba rozwiązanych zagadek
    \item średni czas rozgrywki oraz łączny czas rozgrywki
\end{itemize}

\section{Moduły}

\subsection{Sobokan}

\subsection{Generowanie poziomów}
Poziomy będą proceduralnie generowane za pomocą algorytmu opisanego w pracy
naukowej \textit{`Procedural Generation of Sokoban levels'}\footnote{Joshua Taylor 
\& Ian Parberry `Procedural Generation of Sokoban levels', luty 2011, University 
of North Texas} i będą reprezentowane przez dwuwymiarową tablicę liczb całkowitych
reprezentujących typ bloków.

% IMPLEMENTATION:
% ===============

% C with GTK+
% Features:
% ---------

% point system based on time and difficulty of a level
% levels are procedurally generated (refer to the paper)
% movement with arrows / hjkl
% general data displayed in main menu

\end{document}
