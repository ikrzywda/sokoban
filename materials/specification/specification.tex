\documentclass{article}

\begin{document}

\title{Implementacja gry Sokoban w C i GTK+}
\author{Igor Krzywda}

\maketitile

\section{Sokoban}

Gra polega na rozmieszczeniu skrzyń w odpowiednie miejsca w magazynie.
Skrzynie są przemieszczane przez postać kontrolowaną przez gracza według 
następujących zasad:
\begin{itemize}
    \item skrzynie można jedynie pchać
    \item nie można pchnąć jednej skrzyni za pomocą drugiej
    \item ściany są nieruchome
\end{itemize}

\section{Interakcja z użytkownikiem}

Gracz będzie rozpoczynał rozgrywkę na proceduralnie wygenerowanym
poziomie\footnote{Joshua Taylor \& Ian Parberry `Procedural 
Generation of Sokoban levels', luty 2011, University of North Texas} zgodnie
ze wcześniej wybraną trudnością. Trudność będzie zależna od wielkości planszy.
Postać będzie kontrolowana za pomocą przycisków `hjkl' lub strzałek.
Dodatkowymi opcjami będzie cofnięcie ruchu ('u') oraz pauza ('p'). Po
ukończeniu poziomu, czas wraz z trudnością będą zapisane w celu prowadzenia 
następujących statystyk:
\begin{itemize}
    \item najlepszy czas dla każdej trudności
    \item liczba rozwiązanych zagadek
    \item średni czas rozgrywki oraz łączny czas rozgrywki
\end{itemize}

\section{Implementacja}

Gra będzie zaimplementowana w języku C z pomocą biblioteki GTK. Poziomy
będą generowane proceduralnie za pomocą algorytmu opisanego w publikacji 
UNT. ... yada yada
% IMPLEMENTATION:
% ===============

% C with GTK+
% Features:
% ---------

% point system based on time and difficulty of a level
% levels are procedurally generated (refer to the paper)
% movement with arrows / hjkl
% general data displayed in main menu

\end{document}
